% LaTeX resume using res.cls
\documentclass[line,letterpaper]{resume}
\usepackage{scalefnt}
\usepackage{hyperref}
\usepackage[normalem]{ulem}
%Default
%\usepackage[scale=1]{tgschola}
%Kurier
%\usepackage[math]{kurier}
%\usepackage[T1]{fontenc}
%\usepackage{ascii}
%\usepackage[T1]{fontenc}

\usepackage{multirow}
\hypersetup{colorlinks=false, pdfborder={0 0 0}}
\begin{document}

\name{\textbf{Nathan Aclander} \ \ \ \ \ \ \
      \textit{\href{www.aclander.com}{www.aclander.com}} \ \ \ \ \ \ \
      \textit{\href{mailto:nathan.aclander@ucla.edu}{nathan.aclander@ucla.com}} \ \ \ \ \ \ \ \textit{(949) 436-8945}}
\address{\ }
\address{\ }


\begin{resume}
    \vspace{-24pt}
    \section{\uppercase{Education}} {\sl Bachelor of Science,} \/
    Computer Science \& Engineering \\
    University of California, Los Angeles\\

    \vspace{-16pt}

 \section{\uppercase{Skills}}
    {\sl\textbf{Overview:}}\/
	\begin{itemize}
		\item Experience in writing clean, reliable software deployed across hundreds of thousands of machines
		\item Experience in administration and operation of cross-region computer systems
	\end{itemize}
    {\sl\textbf{Languages:}}\/
    Python, Ruby, Java, Bash, C, C++, GO \\
    {\sl\textbf{Frameworks:}}\/
	Ruby on Rails, Rspec, Spring, Junit, Flask, various AWS services\\
    {\sl\textbf{Operating Systems:}}\/
	Administrating and developing on Linux (RHEL/Debian/Arch) and Unix environments at scale\\
    {\sl\textbf{Software:}}\/
	\sout{Vim} Emacs, Git, Mercurial, SVN, Redmine
    \vspace{-6pt}

	\section{\uppercase{Work Experience}}

	{\sl\textbf{\href{https://aws.amazon.com/}{Amazon - AWS}}} \hfill Winter 2015--Present\\
    \emph{SDE I / II } --- At \underline{\href{http://aws.amazon.com/ec2}
	{Amazon EC2}} where I was part of the Kernel and Operating System team. I helped
	develop next generation EC2 instances. I am now working on the EC2 Tools team
    distributed provisioning and configuration systems.

	{\sl\textbf{\href{www.amazon.com}{Amazon}}} \hfill Summer 2014--Fall 2014\\
    \emph{SDE Intern} --- At \underline{\href{https://www.amazon.jobs/en/teams/infosec}
    {Amazon Information Security}} where I developed an internal tool designed to
	enable employees throughout the company to manage preferences regarding the
	security state of their code.

    {\sl\textbf{\href{www.directv.com}{DIRECTV}}} \hfill Summer 2013-- Fall 2013\\
    \emph{Software Engineering Intern}
    ---  Developed memory leak detection tool for custom linux userland tools,
	with cross architecture compatability(ARM, MIPS, X86)

    {\sl\textbf{\href{www.uclaradio.com}{UCLA Radio}}} \hfill Fall 2011-- Fall 2014\\
    \emph{Web Director} --- UCLA Radio website group leader, website administrator

    {\sl\textbf{\href{https://housing.ucla.edu/residence-hall-computing}
	{Student Technology Center}}} \hfill Summer 2011-- winter 2014\\
    \emph{Computer Technician} --- Customer Support Representative,
IT support (software and hardware), network infrastructure support
    \vspace{-6pt}


    \section{\uppercase{Independent Projects}}
	\begin{itemize}
	\item \underline{\href{https://github.com/naclander/uploader}{Uploader}} -
	A tool to help people share files and links.
        \item
        \underline{\href{https://tinyurl.com/la495b3}{Various contributions}}
        to emacs packages and ecosystem.
	\item Front/Back end development for uclaradio.com, including layout design
	      and feature implementation.
	\item Memory-leak detection tool on an always-on embedded MIPS/X86 system.
	\item Experience Building single threaded and multithreaded server software
	      in C, Twisted, and Node.js.
	\item Large scale software development; multithreaded implementation of GZIP
	      in Java. Reimplementation of the Pacman video game.
	\item Implemented a Unix shell understanding a subset of bash syntax
	that is also able to run commands in parallel.
	\item Personal and professional server management on X86/64 and Arm machines,
	varying *nix distributions(Debian, Ubuntu, RHEL, ArchLinux).
	\item Anti-theft Android App development using Google App Engine.
	\item Implemented a RAM disc block device kernel module supporting
	reading/writing from various threads/processes.
	\end{itemize}
    \vspace{-6pt}

    \hfill \emph{For more projects and contributions see
    \underline{\href{https://github.com/naclander}{naclander on github}}}.

    \vspace{-15pt}

\section{\uppercase{Extra-Curricular}}
    {\sl\textbf{\href{http://linux.ucla.edu}{Linux Users Group}}} \hfill Fall 2013 -- Spring 2014\\
    \emph{Vice President} --- A student organization at UCLA to help promote
	and familiarize other students with the installation and usage of Linux
	operating systems and other open source projects, both for school and for personal use.


   \end{resume}
\end{document}
