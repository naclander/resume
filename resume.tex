% LaTeX resume using res.cls
\documentclass[line,letterpaper]{resume}
\usepackage{scalefnt}
\usepackage{hyperref}
\usepackage[scale=1]{tgschola}
\usepackage{multirow}
\hypersetup{colorlinks=false, pdfborder={0 0 0}}

\begin{document}

\name{Nathan Aclander}
\address{\href{www.aclander.com}{Personal website --- www.aclander.com}}
\address{\href{mailto:nathan.aclander@ucla.com}{nathan.aclander@ucla.com} --- (949) 436-8945}


\begin{resume}
    \vspace{-24pt}
    \section{\uppercase{Education}} {\sl Bachelor of Science,} \/
    Computer Science \& Engineering \\
    University of California, Los Angeles\\
    Graduating, December 2014 \\

    \vspace{-16pt}

	\section{\uppercase{Work Experience}} {\sl\textbf{Amazon}} \hfill
    Summer 2014--Fall 2014\\
    \emph{SDE Intern} --- Worked at \underline{\href{http://security.amazon-jobs.com/index.html}
    {Amazon Information Security}} supporting an internal tool designed to 
	enable employees throughout the company manage preferences regarding the 
	security state of their code.

    {\sl\textbf{DIRECTV}} \hfill Summer 2013-- Fall 2013\\
    \emph{Software Engineering Intern}
    ---  Developed memory leak detection tool for 100k+ allocations,
	cross architecture computability (ARM, MIPS, X86)

    {\sl\textbf{UCLA Radio}} \hfill Fall 2011-- Fall 2014\\
    \emph{Web Director} --- UCLA Radio website group leader, website administrator

    {\sl\textbf{Student Technology Center}} \hfill Summer 2011-- winter 2014\\
    \emph{Computer Technician} --- Customer Support Representative,
IT support (software and hardware), network infrastructure support
    \vspace{-6pt}

	\section{\uppercase{Related Courses}}
	\begin{tabular}{l l l}
	Database Systems & Computer Architecture & Operating Systems  Principles \\
	Algorithm Design & Digital Electronic Circuits & Digital Systems Design \\
	Computer Networks & Programming Languages Principles& Embedded Systems Design\\
    Parallel \& Distributed Computing & Computer Graphics\\
	\end{tabular}

\section{\uppercase{Course and Independent Projects}}
	\begin{itemize}
	\item Front/Back end development for uclaradio.com; layout design, feature implementation
	\item Developed a tool to detect memory leaks on an always-on embedded MIPS/X86 system
	\item Anti-theft Android App development using Google App Engine
	\item Large scale software development; multithreaded implementation of GZIP in Java. Recreation of Pacman video game
	\item Experience Building single threaded and multithreaded server software in C, Twisted, and Node.js
	\item Experience writing a Unix shell understanding a subset of bash syntax that is also able to run commands in parallel
	\item Personal Server management on X86/64 and Arm machines, varying OSs from Debian to CentOS
	\item Implemented a RAM disc block device kernel module supporting reading/writing from various threads/processes
	\item Implemented a simple file system supporting file, directories, soft/hard and conditional links
	\item Designed a p2p client that supported parallel upload/downloads while maintaining OS security
	\item Experience in analog and digital circuit design using OrCAD and Cadence
	\item Particle Cell Code optimization using CUDA for C extension library
	\end{itemize}
    \vspace{-6pt}

    \hfill \emph{For more projects see
    \underline{\href{https://github.com/naclander}{naclander on github}}}.

    \vspace{-15pt}

    \section{\uppercase{Skills}}
    {\sl\textbf{Languages:}}\/
    C, C++, ruby (and Rails), bash, Coffeescript (and Jquery), Java, Python\\
    {\sl\textbf{Operating Systems:}}\/
	Extended experience developing on Linux environments, experience developing on
	MS Visual Studio, XCode \\
    {\sl\textbf{Software:}}\/
	\begin{itemize}
	\item Experience in Code Revision Software using Git, Mercurial, CVS, 
	\item Experience with bug reporting software such Redmine and JIRA.
	\end{itemize}
    \vspace{-6pt}

    
    \section{\uppercase{Extra-Curricular}}
    {\sl\textbf{Linux Users Group}} \hfill Fall 2013 -- Spring 2014\\
    \emph{Vice President} --- A student organization at UCLA to help promote
	and familiarize other students with the installation and usage of Linux,
	both for school and for personal use.

\end{resume}
\end{document}
